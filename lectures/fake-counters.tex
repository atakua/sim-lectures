\input{../common/config}

\author{Grigory Rechistov \thanks{grigory.rechistov@intel.com}}
\title{Balance between fidelity and usefulness of simulation}
\subtitle{Case study of fake counters}
\institute{Intel Corporation}

\begin{document}

\startslides

\section{Hardware counters}

\begin{frame}{Simulation of hardware counters}

\begin{quotation}
Essentially, all models are wrong, but some are useful

G. E. P. Box, and N. R. Draper, Empirical Model Building and Response Surfaces (1987). 
\end{quotation}


\begin{itemize}
\item A counter — register that increments when certain event happens
\item Examples: instruction counter, time counter, cache misses
\item Almost "counters": temperature, pipeline occupancy
\end{itemize}

In our context, they all have a method to "read" them (and sometimes to write them).
Sometimes they are monotonic, sometimes not

\end{frame}

\section{Problem statement}

\begin{frame}{Why going after the most correct solution is not always a good idea}

\begin{enumerate}

\item No internal microarchitectural details to produce the events. Example: cache misses

\item No external data to produce events. Example: power and temperature 
  
\item Missing functionality in an external data provider. Example: residency counters of sleep states
  
\item Slow to simulate. Example: taken branch counter

\end{enumerate}

\end{frame}

\section{Spectrum of solutions}

\begin{frame}{Spectrum of solutions}

\begin{enumerate}
\item No counter
\item Always zero
\item Constant non-zero
\item Monotonically running fixed rate
\item Pseudorandom 
\item User-supplied static value
\item Honest simulation 
\item Choice between honest and simplified models
\end{enumerate}

\end{frame}

\begin{frame}{No counter}

Target software is simply not allowed to access a particular register, and if it does, it gets an architectural exception. 
  
Such implementation should be accompanied with a correct "absent" enumeration of the register.

Correctly written target software should be able to properly discover absence of a counter and not attempt to access it. Might not be vey useful though

\end{frame}

\begin{frame}{Always zero}

A register is present but always returns zero. 

Target software won't crash on accessing the register. However, it might become confused when/if
it detects that the values read from the counter are not changing. 

Useful if no decisions are made based on the absolute values of the register. E.g. if it is only printed.

\end{frame}


\begin{frame}{Constant non-zero}

Zero is often a special number. It might cause problems for target software not expecting it.

Always return constant non-zero value.

Examples: temperature reading registers, power consumption, frequency.

Useful if no decisions are made based on comparisons of values taken at different times. 

\end{frame}
  
\begin{frame}{Monotonic}

Monotonically running counter with fixed rate. 

Sort of a the golden middle. 

All possible events in a target usually correlate with program's progress, which in turn correlates
with both cycle and step counters.

A notion of time (step, cycle counter) is almost always present in a simulation.
  
Returning current cycle count (optionally scaled by a constant rate) is often good enough for a target software to believe that
things works properly.

Examples: almost all performance counters connected to target program flow.  
  
\end{frame}
  
\begin{frame}{Pseudrandom}

Certain applications might become suspicious if the rate counter is too "constant". 

Or maybe users want to test their code with more variability in data. 

Drive the counter with a flow of pseudo-random numbers.

For monotonically increasing counters, the rate of their change may also be
modulated by randomness.
  
Examples: RNG registers, microarchitectural telemetrics (such as pipeline load factors)

\end{frame}
  
\begin{frame}{User-supplied values}

Give the control over counters to those who need them: to the model user.

Provide programmable/scriptable interface to feed the model with counter data from outside.

A lot of flexibility, but it comes at the cost of shifting the burden of implementation to a user.

Still, a very good way to build a plugin-based software model.

\end{frame}

\begin{frame}{(Partially) honest simulation}
In certain cases there is no better alternative than to just implement a counter right. 

There are still often possibility to simplify an implementation while keeping it useful.
  
Example: registers meant to count \textit{mispredicted} branch instructions.
By reporting \textit{all} branches as mispredicted, it is still useful to software expecting
valid data, but the model still does not sport an actual branch predictor unit model.

\end{frame}
  
\begin{frame}{Configurable choice}

Provide a configuration knob to control which type of model is active: slow and accurate or
fast and totally fake.

An honest register may be expensive and would slow the whole simulation down.

It is reasonable to enable it only for those cases when target software really needs its. 

Example: Last Branch Record (LBR) registers are meant to keep
recent history of source and target instruction pointers for branch
instructions. Tracking all instructions to detect branches hinders many simulation optimization possibilities.

Target software rarely uses LBRs. Simulating them should be avoided unless absolutely needed.
  
\end{frame}

\section{Let's generalize}

\begin{frame}{Why does faking work?}

Note how all of these counters are mostly read-only. The data used to produce the result is not directly observable by the receiver:

$$observable\_result = F(hidden\_state)$$

Target software has no direct visibility into hidden state and thus cannot "feel" the lack of fidelity in returned results

\end{frame}

\begin{frame}{When will faking not work?}

Reply-response transactions are much harder to fool with fake data

Compare with faking result of addition: $A = B + C$. In this case, $B$ and $C$ are under control of the requester.

$$result = F(input state)$$

Sometimes inputs are combination of known and hidden:

$$result = F(input state, hidden state)$$


\end{frame}


\section{Conclusions}

\begin{frame}{Summary}
\begin{itemize}
\item Honest simulation is often expensive
\item But it is not always needed!
\item We can cut corners for unidirectional transactions
\item A spectrum of strategies to choose from
\item The choice should be driven by customer scenarios
% \item 
\end{itemize}
\end{frame}

% \begin{frame}[allowframebreaks]{Bibliography}
% \begin{thebibliography}{99}
  % \bibitem{} \textit{todo authors}. TODO title

% \end{thebibliography}
% \end{frame}

\finalslide

\end{document}
