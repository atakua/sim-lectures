\input{../common/config}

\author{Grigory Rechistov \thanks{grigory.rechistov@intel.com}}
\title{Simulation of cluster systems}
\institute{Intel Corporation}

\begin{document}

\startslides

\begin{frame}{About the Presenter}
\begin{itemize}
\item At MIPT: 2004−2016
\item At Intel: 2007−2021
\item Read this course in 2010-2016
\end{itemize}
\end{frame}

\begin{frame}{Ice breaker}

What is the biggest system you managed to (accidentally) bring down?

\end{frame}


\section{Problem definition}

\begin{frame}{What is a cluster?}

\pause
\begin{quotation}
A distributed system is one in which the failure of a computer you didn't even know existed can render your own computer unusable

Leslie Lamport
\end{quotation}
\pause

A group of computers connected by network.

\begin{itemize}
    \item A cluster is usually managed by a single organization
    \item Usually uniform structure
    \item Reliability is less important than raw performance
\end{itemize}

\end{frame}

\begin{frame}{Simulation of large computer systems}

\begin{enumerate}
    \item Modern computer applications include large networks of individual computers
    \item Similarly to simulation of individual systems, there may be value in simulation of networks of systems
    \item Apparent challenge: target won't fit into a host
\end{enumerate}

\end{frame}

\begin{frame}{What problems does such simulation solve?}

Possible issues:

\begin{enumerate}
    \item Study of scalability effects/limitations
    \item Tracing of communication patterns
    \item Development of cluster software
\end{enumerate}

\end{frame}

\begin{frame}{What about the challenge?}

Let's try the obvious solution: distribute the simulation over a network of host systems.

We now have a target network of target nodes represented by a host network of host nodes.

\end{frame}


\section{Components of solution}

\begin{frame}{What constitutes the solution}

\begin{itemize}
    \item Models of individual target compute nodes
    \item Model of target network (links and routers)
    \item Means for time synchronization
    \item Framework to distribute the simulation
    \item Framework to manage the simulation
\end{itemize}

\end{frame}

\begin{frame}{Target nodes}

\begin{itemize}
\item Models of a single computer system
\item Typically consists of: CPUs, memory, storage, network card
\item Typically runs: target software stack: EFI, hypervisor, OS, user application
\end{itemize}

\end{frame}

\begin{frame}{Target network}

\begin{itemize}
\item Physical channels (Ethernet, Infiniband, etc.)
\item Models of routing devices
\item Routers may be also computers with own software
\item Or they may be abstracted as black boxes with no internal structure \pause
\item …can compute nodes be black boxes too‽ Why not!
\end{itemize}

\end{frame}


\begin{frame}{Global target time}

\begin{itemize}
\item Surprisingly hard topic. Required to maintain consistent target state 
\item Examples of algorithms: naive host time pass-through, global barrier, time warp protocol… 
\item Remember: we should be able to synchronously start and resume the simulation, transparently to the target
\end{itemize}

\end{frame}

\begin{frame}{Framework to distribute simulation}

\begin{itemize}
\item More than one target node can fit into a single target host
\item Targets on a single host may be running in parallel if host is multiprocessor
\item Across multiple hosts, the simulator framework should provide a transparent mechanism to communicate (pass target network packets) and coordinate (target time, start/stop etc.) groups of target nodes
\end{itemize}

\end{frame}

\begin{frame}{Break time}
\end{frame}


\begin{frame}{But how do they fit?}

\begin{itemize}
\item Two principles: 1) sharing is caring; 2) being lazy is a virtue
\item Target nodes are likely to run identical software. Read-only parts of memory are shared.
\item Target software may use a limited amount of resources actively at any moment
\end{itemize}

Examples: 100 GByte target disk may occupy only 2 GB of host space. RAM of 10 target systems  may be deduplicated into working set of about 1 target system.

\end{frame}


\begin{frame}{Managing the simulation}

Imagine yourself standing in a room full of servers. How'd you approach…

\begin{itemize}
\item Starting all of them?
\item Distributing the application task?
\item Controlling and inspecting states of individual components?
\item Shutting them down?
\end{itemize}

Obviously, a separate messaging network is required to be provided by the simulation framework to support these activities in a centralized manner

\end{frame}


\section{Practical application}

\begin{frame}{Simulation of 10000 processor cores}

\Huge HERE BE PICTURES

\end{frame}



\section{End}

\begin{frame}{Conclusions}
\begin{itemize}
\item Distributed simulation may be a right tool for the right task
\item It requires substantial simulation framework support 
\item However, many goals may often be achieved by more modest means
\item This is the science for you: not all hypotheses turn out to be true
\end{itemize}
\end{frame}

\begin{frame}[allowframebreaks]{Bibliography}
\begin{thebibliography}{99}

  \bibitem{itj} \textit{Grigory Rechistov} Simics* on Shared Computing Clusters: the Practical Experience of Integration and Scalability \url{https://atakua.org/w/images/Intel\%20ITJ60\%2011-15-2013.pdf}
  
  \bibitem{progeng2012} \textit{Г.С. Речистов and А.А. Иванов and П.Л. Шишпор and В.М. Пентковский} Моделирование компьютерного кластера на распределённом симуляторе. Верификация моделей вычислительных узлов и сети кластера \url{https://atakua.org/w/images/printedl\%20pi612_web-24-29.pdf}
  \bibitem{issc2012} \textit{G.S. Rechistov and A.A. Ivanov and P.L. Shishpor and V.M. Pentkovski} Simulation and Performance Study of Large Scale Computer Cluster Configuration: Combined Multi-level Approach \url{http://www.sciencedirect.com/science/article/pii/S1877050912002049}  

\end{thebibliography}
\end{frame}

\begin{frame}{On the Next Lecture}
?
\end{frame}

\finalslide

\end{document}
