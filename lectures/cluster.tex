\input{../common/config}

\author{Grigory Rechistov \thanks{grigory.rechistov@intel.com}}
\title{Simulation of cluster systems; limits of simulation}
\institute{Intel Corporation}

\begin{document}

\startslides

\begin{frame}{About the Presenter}
\begin{itemize}
\item At MIPT: 2004−2016
\item At Intel: 2007−2021
\item Read this course in 2010-2016
\end{itemize}
\end{frame}

\begin{frame}{Ice breaker}

todo

\end{frame}


\section{Problem definition}


\begin{frame}{What is a cluster?}

\begin{quotation}
A distributed system is one in which the failure of a computer you didn't even know existed can render your own computer unusable

Leslie Lamport
\end{quotation}

\begin{itemize}
    \item A cluster is usually managed by a single organization
    \item Usually more uniform structure
    \item Reliability is less important than raw performance
\end{itemize}

\end{frame}

\begin{frame}{Simulation of large computer systems}

\begin{enumerate}
    \item Modern computer applications include large networks of individual computers
    \item Similarly to simulation of individual systems, there may be value in simulation of networks of systems
\end{enumerate}

\end{frame}

\begin{frame}{What problem does it solve?}

\begin{enumerate}
    \item Study of scalability effects
    \item Tracing of remote
    \item Tracing of remote
    \item 
\end{enumerate}

\end{frame}



\section{Components of solution}

\begin{frame}{todo}


\begin{enumerate}
    \item 
\end{enumerate}

\end{frame}

\section{Practical application}

\begin{frame}{Simulation of 10000 processors}


\begin{enumerate}
    \item 
\end{enumerate}

\end{frame}


\section{End}

\begin{frame}{Summary}
\begin{itemize}
\item 
\end{itemize}
\end{frame}

\begin{frame}[allowframebreaks]{Bibliography}
\begin{thebibliography}{99}
  \bibitem{itj} \textit{Grigory Rechistov} Simics* on Shared Computing Clusters: the Practical Experience of Integration and Scalability \url{https://atakua.org/w/images/Intel%20ITJ60%2011-15-2013.pdf}
  \bibitem{progeng2012} \textit{Г.С. Речистов and А.А. Иванов and П.Л. Шишпор and В.М. Пентковский} Моделирование компьютерного кластера на распределённом симуляторе. Верификация моделей вычислительных узлов и сети кластера \url{https://atakua.org/w/images/printedl%20pi612_web-24-29.pdf}
  \bibitem{issc2012} \textit{G.S. Rechistov and A.A. Ivanov and P.L. Shishpor and V.M. Pentkovski} Simulation and Performance Study of Large Scale Computer Cluster Configuration: Combined Multi-level Approach \url{http://www.sciencedirect.com/science/article/pii/S1877050912002049}  

\end{thebibliography}
\end{frame}

\begin{frame}{On the Next Lecture}
?
\end{frame}

\finalslide

\end{document}
